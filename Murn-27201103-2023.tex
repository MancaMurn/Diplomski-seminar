\documentclass[mat1, tisk]{fmfdelo}
\usepackage{amsmath}
\usepackage{graphicx}
\usepackage{hyperref}
%\usepackage[maxbibnames=99, articlein=false]{biblatex}
%\addbibresource{literatura.bib}
% \documentclass[fin1, tisk]{fmfdelo}
% Če pobrišete možnost tisk, bodo povezave obarvane,
% na začetku pa ne bo praznih strani po naslovu, …

%%%%%%%%%%%%%%%%%%%%%%%%%%%%%%%%%%%%%%%%%%%%%%%%%%%%%%%%%%%%%%%%%%%%%%%%%%%%%%%
% METAPODATKI
%%%%%%%%%%%%%%%%%%%%%%%%%%%%%%%%%%%%%%%%%%%%%%%%%%%%%%%%%%%%%%%%%%%%%%%%%%%%%%%

% - vaše ime
\avtor{Manca Murn}

% - naslov dela v slovenščini
\naslov{Metrična dimenzija leksikografskega produkta grafov}

% - naslov dela v angleščini
\title{The metric dimension of the lexicographic product of graphs}

% - ime mentorja/mentorice s polnim nazivom:
%   - doc.~dr.~Ime Priimek
%   - izr.~prof.~dr.~Ime Priimek
%   - prof.~dr.~Ime Priimek
%   za druge variante uporabite ustrezne ukaze
\mentor{Sandi Klavžar}
% \somentor{...}
% \mentorica{...}
% \somentorica{...}
% \mentorja{...}{...}
% \somentorja{...}{...}
% \mentorici{...}{...}
% \somentorici{...}{...}

% - leto diplome
\letnica{2023} 

% - povzetek v slovenščini
%   V povzetku na kratko opišite vsebinske rezultate dela. Sem ne sodi razlaga
%   organizacije dela, torej v katerem razdelku je kaj, pač pa le opis vsebine.
\povzetek{...}

% - povzetek v angleščini
\abstract{... }

% - klasifikacijske oznake, ločene z vejicami
%   Oznake, ki opisujejo področje dela, so dostopne na strani https://www.ams.org/msc/
\klasifikacija{..., ...}

% - ključne besede, ki nastopajo v delu, ločene s \sep
\kljucnebesede{...\sep ...}

% - angleški prevod ključnih besed
\keywords{...\sep ...} % angleški prevod ključnih besed

% - angleško-slovenski slovar strokovnih izrazov
\slovar{
% \geslo{angleški izraz}{slovenski izraz}
% ...
}

% - ime datoteke z viri (vključno s končnico .bib), če uporabljate BibTeX
% \literatura{....bib}

%%%%%%%%%%%%%%%%%%%%%%%%%%%%%%%%%%%%%%%%%%%%%%%%%%%%%%%%%%%%%%%%%%%%%%%%%%%%%%%
% DODATNE DEFINICIJE
%%%%%%%%%%%%%%%%%%%%%%%%%%%%%%%%%%%%%%%%%%%%%%%%%%%%%%%%%%%%%%%%%%%%%%%%%%%%%%%

% naložite dodatne pakete, ki jih potrebujete
% \usepackage{...}

% deklarirajte vse matematične operatorje, da jih bo LaTeX pravilno stavil
% \DeclareMathOperator{\...}{...}

% vstavite svoje definicije ...
% \newcommand{\...}{...}
\newcommand{\1}{(1, 1, ..., 1)}
\newcommand{\2}{(2, 2, ..., 2)}



%%%%%%%%%%%%%%%%%%%%%%%%%%%%%%%%%%%%%%%%%%%%%%%%%%%%%%%%%%%%%%%%%%%%%%%%%%%%%%%
% ZAČETEK VSEBINE
%%%%%%%%%%%%%%%%%%%%%%%%%%%%%%%%%%%%%%%%%%%%%%%%%%%%%%%%%%%%%%%%%%%%%%%%%%%%%%%

\begin{document}

\section{Uvod}
V decembru leta 2010 sta v razmaku 17 dni nastala dva različna članka z enakim naslovom - 
\textit{''The metric dimension of the lexicographic product of graph''}. Avtorji obeh člankov 
niso vedeli za delo drugega in so se teme lotili na dva posvem različna načina. V tem diplomskem 
seminarju si bomo ogledali pojma metrične in sosedske dimenzije grafa in njune osnovne lastnosti, 
ter povezave med njima, definirali bomo leksikografski produkt grafov ter povzeli 
glavne rezultate o metrični dimenziji leksikografskega produkta iz obeh člankov.


%%%%%%%%%%%%%%%%%%%%%%%%%%%%%%%%%%%%%%%%%%%%%%%%%%%%%%%%%%%%%%%%%%%%%%%%%%%%%%%
%%%%%%%%%%%%%%%%%%%%%%%%%%%%%%%%%%%%%%%%%%%%%%%%%%%%%%%%%%%%%%%%%%%%%%%%%%%%%%%


\subsection{Osnovni pojmi} \label{ss:osnovni_pojmi}
Za začetek ponovimo nekaj osnovnih definicij in oznak iz teorije grafov, ki jih bomo potrebovali 
za razumevanje tega diplomskega seminarja. 

\begin{definicija} \label{def:graf}
    Graf $G$ je urejen par $(V(G), E(G)),$ kjer je $V(G)$ množica vozlišč in $E(G)$ 
    podmnožica v $\binom{V(G)}{2},$ ki vsebuje povezave grafa.
\end{definicija}

Če je $V(G)$ končna množica, je $G$ končen graf. Število $|V(G)|$ imenujemo red grafa. 
Če je med dvema različnima vozliščema največ ena povezava in nobeno vozlišče ni povezano samo 
s seboj, pravimo, da je graf enostaven. Povezave med vozlišči $\{u, v\}$ bomo zaradi preglednosti
pisali kar $uv$. Vozlišči $v, u \in G$ sta sosedni, če $uv \in E(G).$ 
Sosednost je ekvivalenčna relacija, zato sosedni vozlišči označimo $u \sim v.$ Če $w, x \in V(G)$ 
nista sosedni pa pišemo $w \not \sim x.$

\begin{definicija} \label{def:sosescina}
    Naj bo $G$ graf in $v \in V(G)$. Množico 
    $$N(v) = \{u \in V(G) \, | \,vu \in E(G) \}$$ imenujemo soseščina vozlišča $v$.

    Stopnja vozlišča je ${\rm deg}(u) = |N(u)|.$
\end{definicija}


\begin{definicija} \label{def:komplement}
    Komplement grafa $G$, je graf $\overline{G},$ za katerega velja $V(G) = V(\overline{G})$ in 
    $$\forall u,v \in V(\overline{G}): uv E(\overline{G}) \Leftrightarrow uv \not \in E(G).$$
\end{definicija}

Sprehod v grafu $G$ je zaporedje vozlišč $v_1, v_2, ... v_k$ iz $V(G)$, tako da je 
$\forall i : v_i, v_{i+1} \in E(G).$ Sprehod je enostaven, če vsebuje sama različna vozlišča.
Graf je povezan, če med vsakima dvema različnima vozliščema obstaja sprehod. Na povezanem
grafu lahko definiramo razdaljo med vozliščema.

\begin{definicija} \label{def:razdalja}
    Razdalja med dvema vozliščema $u, v \in V(G)$ je dolžina najkrajšega sprehoda in jo 
    označujemo z $d_{G}(u, v).$ 
\end{definicija}

Naslednja trditev o razdaji med vozlišči je očitna.

\begin{trditev} \label{trd:nicelna_razdalja}
    Za povezan graf $G$ in poljubni vozlišči $v, w \in V(G)$ velja:
    $$ d_{G}(v, w) = 0 \Leftrightarrow v=w.$$
\end{trditev} 

\begin{definicija} \label{def:premer}
    Premer povezanega grafa $G$ označujemo z ${\rm diam}(G)$ in je enak največji razdalji med vozlišči.
    Torej $${\rm diam}(G) = \underset{v, u \in V(G)}{\max} d_{G}(u, v).$$
\end{definicija}

Iz definicije očitno sledi, da za poljubni dve vozlišči $u, v$ iz povezanega grafa $G$
velja $0 \leq d_G (u, v) \leq {\rm diam}(G).$

\begin{definicija} \label{def:podgraf}
    Graf $H$ je podgraf grafa $G$, če velja $V(H) \subseteq V(G)$ in 
    $E(H) \subseteq E(G)$.

    Podgraf $H$ je induciran, če velja 
    $\forall u, v \in V(H) : uv \in E(G) \Rightarrow E(H)$.
\end{definicija}

\begin{definicija} \label{def:komponenta}
Komponenta grafa je povezan podgraf, ki ni del nobenega večjega povezanega podgrafa. 
\end{definicija}
%nerazumljiv komentar

Povezan graf ima seveda samo eno komponento. Definirajmo še operacijo spojitve grafov.

\begin{definicija} \label{def:spoj}
    Spoj grafov $G$ in $H$, je graf $G + H$, za katerega velja $V(G + H) = V(G) \cup V(H)$ 
    in $E(G + H)  = E(G) \cup E(H) \cup \{ uv \;  | \;  u \in V(G) \land v \in V(H) \}.$
\end{definicija}

Poglejmo še nekaj primerov osnovnih razredov grafov:
\begin{itemize} \label{razredi_grafov}
    \item Graf brez povezav na $n$ vozliščih, ki ga označujemo z $N_n$, nima nobenih povezav. 
    \item Polni graf na $n$ vozliščih, ki ga označujemo s $K_n$, ima vse možne povezave.
    \item Polni dvodelni graf $K_{n, m}$ ima množico 
    vozlišč $V(K_{n,m}) = \{ v_{1, 1}, v_{1, 2}, ... , v_{1, n},  \allowbreak v_{2, 1}, v_{2, 2}, ... , v_{2, m} \}$
    in povezave $E(K_{n, m}) = \{ v_{1, i} v_{2, j} \; | \; i, j \in \{ 1, 2, ... , m \} \}.$ 
    \item  Polni $t$-delni graf, označen s $K_{m_1, ..., m_t}$, ima množico 
    vozlišč $V(K_{m_1, ..., m_t}) = \{ v_{1, 1}, v_{1, 2}, ... , v_{1, m_1}, 
     v_{2, 1}, v_{2, 2}, ... , v_{2, m_2}, ... , v_{t, 1}, v_{t, 2}, ... , v_{t, m_t}\}$,
    množica povezav pa je $E(K_{m_1, ..., m_t}) = \{  v_{a, i} v_{b, j} \; | \; a \neq b \land \; 
    i, j \in \{ 1, 2, ... , m \} \}.$
    \item Zvezda na $n$ vozliščih je poseben primer polnega dvodelnega grafa in jo označujemo
    s $S_{n-1} = K_{1, n-1}$
    \item Pot na $n$ vozliščih, ki jo označujemo s $P_n$, ima množico povezav 
    $E(P_n) = \{ v_1 v_2 , v_2 v_3 , ... , v_{n-1} v_n\}.$
    \item Cikel na $n$ vozliščih, dobimo tako, da grafu $P_n$ dodamo povezavo $v_n v_1$. Označimo ga s $C_n$.
    \item Polni razcepljeni graf na $k+l$ vozliščih je enak spoju poti in grafa brez povezav ter ga 
    označujemo s $F_{k,l} = N_k + P_l.$
    %fan graph
    \item Drevo je povezan graf, ki ne vsebuje nobenega cikla.
\end{itemize}

\begin{opomba}
    Očitno velja $P_1 = K_1 = N_1.$ To je graf s samo enim vozliščem. Običajno ga bomo
    označevali s $K_1.$
\end{opomba}


%%%%%%%%%%%%%%%%%%%%%%%%%%%%%%%%%%%%%%%%%%%%%%%%%%%%%%%%%%%%%%%%%%%%%%%%%%%%%%%
%%%%%%%%%%%%%%%%%%%%%%%%%%%%%%%%%%%%%%%%%%%%%%%%%%%%%%%%%%%%%%%%%%%%%%%%%%%%%%%
%%%%%%%%%%%%%%%%%%%%%%%%%%%%%%%%%%%%%%%%%%%%%%%%%%%%%%%%%%%%%%%%%%%%%%%%%%%%%%%
%%%%%%%%%%%%%%%%%%%%%%%%%%%%%%%%%%%%%%%%%%%%%%%%%%%%%%%%%%%%%%%%%%%%%%%%%%%%%%%


\section{Metrična dimenzija grafa} \label{s:mdim}

%motivacija neki izvirnega
TODO - motivacija

%%%%%%%%%%%%%%%%%%%%%%%%%%%%%%%%%%%%%%%%%%%%%%%%%%%%%%%%%%%%%%%%%%%%%%%%%%%%%%%
%%%%%%%%%%%%%%%%%%%%%%%%%%%%%%%%%%%%%%%%%%%%%%%%%%%%%%%%%%%%%%%%%%%%%%%%%%%%%%%


\subsection{Definicija} \label{ss:def_mdim}

Metrična dimenzija grafa je najmanjše število vozlišč grafa, ki jih potrebujemo, da
vsa vozlišča v grafu razlikujemo med sabo zgolj s pomočjo razdalj do izbranih vozlišč.
Formalno to povemo takole:

\begin{definicija} \label{def:mdim}
    Naj bo $G$ povezan graf. 
    \begin{itemize}
        \item Naj bo $W = \{ w_1, ... , w_k  \} \subseteq V(G)$ neprazna podmnožica vozlišč. 
        Vektor $r_W(v) = (d(v, w_1), ..., d(v, w_k))$ imenujemo metrična 
        predstavitev vozlišča $v \in V(G)$ s podmnožico $W$.
        \item Neprazna podmnožica $R \subseteq V(G)$ je rešljiva,
        če $\forall u, v \in V(G): u \neq v \implies r_R(v) \neq r_R(u)$.
        \item Najmanjša rešljiva množica grafa $G$ se imenuje metrična baza. Njeno velikost imenujemo 
        metrična dimenzija in jo označimo z $\beta(G).$
    \end{itemize}
\end{definicija}

Za lažje razumevanje si poglejmo nekaj lahkih osnovnih primerov.

\begin{primer} \label{pr:mdim_pot}
Označimo vozlišča poti z $v_1, v_2, ..., v_n$, kot je prikazano na spodnji sliki \ref{fig:pot}. 
Izberimo podmnožico $W = \{v_1\} \subseteq V(G).$ Metrične predstavitve vozlišč grafa $P_n$, 
glede na $W$, so potem sledeče:
\begin{align*}
    r_W(v_1) = d(v_1, v_1) & = 0 \\
    r_W(v_2) = d(v_2, v_1) & = 1 \\
    & \dots \\
    r_W(v_{n-1}) = d(v_{n-1}, v_1) & = n-2 \\
    r_W(v_n) = d(v_n, v_1) & = n-1.
\end{align*}

Vidimo, da so metrične predstavitve vseh vozlišč med seboj različne. Sledi, da je $W$ 
rešljiva množica. Ker je njena velikost enaka $1$ in je to najmanjša možna neprazna podmnožica 
vozlišč, je torej metrična dimenzija grafa poti poljubne dolžine enaka $\beta(P_n) = 1.$

\begin{figure}[h]
    \centering
    \includegraphics[width=0.6\textwidth]{IMG_pot.jpg}
    \caption{Graf $P_5$}
    \label{fig:pot}
\end{figure}

\end{primer}


\begin{primer}\label{pr:mdim_cikel}
    Označimo vozlišča cikla z $v_1, v_2, ..., v_n$, kot je prikazano na sliki \ref{fig:cikel}. 
    Izberimo podmnožico $W = \{ v_1, v_2 \} \subseteq V(G).$ Metrične predstavitve vozlišč grafa $C_n$, 
    glede na $W$, so potem sledeče:
    \begin{align*}
        r_W(v_1) = (d(v_1, v_1), d(v_1, v_2)) & = (0, 1) \\
        r_W(v_2) = (d(v_2, v_1), d(v_2, v_2)) & = (1, 0) \\
        & \dots \\
        r_W(v_{n-1}) = (d(v_{n-1}, v_1), d(v_{n-1}, v_2)) & = (2, 3) \\
        r_W(v_n) = (d(v_n, v_1), d(v_n, v_2)) & = (1, 2)
    \end{align*}
    
    Zopet vidimo, da so metrične predstavitve vseh vozlišč med seboj različne. Če bi vzeli 
    množico s samo enim vozliščem, bi imeli po dve vozlišči enako metrično prestavitev.
    $W$  je torej najmanjša rešljiva množica, njena velikost pa je enaka $2$. Metrična 
    dimenzija poljubno velikega cikla je enaka $\beta(C_n) = 2.$

    \begin{figure}[h]
        \centering
        \includegraphics[width=0.4\textwidth]{IMG_cikel.jpg}
        \caption{Graf $C_5$.}
        \label{fig:cikel}
    \end{figure}

\end{primer}


\begin{primer}\label{pr:mdim_poln}
    Označimo vozlišča polnega grafa z $v_1, v_2, ..., v_n$, kot je prikazano na sliki \ref{fig:polni}. 
    Izberimo podmnožico $W = \{ v_1, v_2, ... , v_{n-1} \} \subseteq V(G).$ 
    Metrične predstavitve vozlišč grafa $K_n$, glede na $W$, so potem sledeče:
    \begin{align*}
        r_W(v_1) = (d(v_1, v_1), d(v_1, v_2), ... , d(v_1, v_{n-1})) & = (0, 1, ... , 1) \\
        r_W(v_2) = (d(v_2, v_1), d(v_2, v_2), ... , d(v_2, v_{n-1})) & = (1, 0, ... , 1) \\
        & \dots \\
        r_W(v_{n-1}) = (d(v_{n-1}, v_1), d(v_{n-1}, v_2), ... , d(v_{n-1}, v_{n-1})) & = (1, 1, ... , 0) \\
        r_W(v_n) = (d(v_n, v_1), d(v_n, v_2), ... ,  d(v_n, v_{n-1})) & = (1, 1, ... , 1)
    \end{align*}
    
    Zopet vidimo, da so metrične predstavitve vseh vozlišč med seboj različne. Vsako 
    vozlišče ima na $i$ - ti komponenti metrične predstavitve $0$ in povsod drugje $1$, 
    z izjemo vozlišča $v_n$, ki ima povsod $1$. Če bi iz $W$ izvzeli poljubno vozlišče $v_i$, 
    bi imeli vozlišči $v_i$ in $v_n$ enaki metrični predstavitvi. $W$  je torej najmanjša rešljiva množica, 
    njena velikost pa je $n-1$. Metrična dimenzija poljubno velikega polnega grafa je enaka 
    $\beta(K_n) = n-1.$

    \begin{figure}[h]
        \centering
        \includegraphics[width=0.4\textwidth]{IMG_polni.jpg}
        \caption{Graf $K_5$.}
        \label{fig:polni}
    \end{figure}

\end{primer}

%%%%%%%%%%%%%%%%%%%%%%%%%%%%%%%%%%%%%%%%%%%%%%%%%%%%%%%%%%%%%%%%%%%%%%%%%%%%%%%
%%%%%%%%%%%%%%%%%%%%%%%%%%%%%%%%%%%%%%%%%%%%%%%%%%%%%%%%%%%%%%%%%%%%%%%%%%%%%%%

\subsection{Sosedska dimenzija grafa} \label{ss:sdim}
V nekaterih primerih si bomo pri obravnavanju metrične dimenzije pomagali tudi s pojmom 
sosedske dimenzije. 

Pri sosedski dimenziji zopet iščemo podmnožico vozlišč, s pomočjo katerih bomo lahko vsa 
vozlišča v grafu med sabo razlikovali, vendar tokrat ne s pomočjo razdalje, pač pa s 
pomočjo relacije sosednosti. 

Definirajmo preslikavo  $a: V(G) \times V(G) \rightarrow \mathbb{N}$ takole:  
\begin{equation} \label{eq:fja_a} 
    a(v, w) = 
    \begin{cases}
        0; & v = w \\
        1; & v \sim w \\
        2; & v \not\sim w
    \end{cases} 
\end{equation} 

Sedaj lahko zapišemo naslednjo definicijo.

\begin{definicija} \label{def:sdim}
    Naj bo $G$ poljuben graf. 
    \begin{itemize}
        \item Naj bo $W = \{ w_1, ... , w_k  \} \subseteq V(G)$ neprazna podmnožica vozlišč. 
        Vektor $s_W(v) = (a(v, w_1), ..., a(v, w_k))$ imenujemo 
        sosedska predstavitev vozlišča $v \in V(G)$ s podmnožico $W$.
        \item  Podmnožica vozlišč $S \subseteq V(G)$ je sosedsko rešljiva,
        če $\forall u, v \in V(G): u\neq v \implies s_S(v) \neq s_S(u)$.
        \item Najmanjša sosedsko rešljiva množica grafa $G$ se imenuje sosedska baza. 
        Njeno velikost imenujemo sosedska dimenzija in jo označimo z $\mu (G).$
    \end{itemize}    
\end{definicija}


%%%%%%%%%%%%%%%%%%%%%%%%%%%%%%%%%%%%%%%%%%%%%%%%%%%%%%%%%%%%%%%%%%%%%%%%%%%%%%%


\subsubsection{Lastnosti sosedske dimenzije} \label{sss:lastnosti_sdim}

\begin{trditev} \label{trd:lastnosti_sdim}
    Naj bo $G$ povezan graf. Potem velja:
    \begin{enumerate}
        \item $\mu(G) \geq \beta(G)$.
        \item ${\rm diam}(G) = 2 \Rightarrow \mu(G) = \beta(G).$
        \item $\mu(G) = \mu(\overline{G}).$
        \item $\mu(G) = 1 \Leftrightarrow G \in \{P_1, P_2, P_3, \overline{P_2}, \overline{P_3}\}.$
        \item $\mu(G) = n - 1 \Leftrightarrow G \in \{K_n, \overline{K_n}\}.$
    \end{enumerate}
\end{trditev}

\begin{dokaz}
    TODO
\end{dokaz}

% primeri
%%%%%%%%%%%%%%%%%%%%%%%%%%%%%%%%%%%%%%%%%%%%%%%%%%%%%%%%%%%%%%%%%%%%%%%%%%%%%%%
%%%%%%%%%%%%%%%%%%%%%%%%%%%%%%%%%%%%%%%%%%%%%%%%%%%%%%%%%%%%%%%%%%%%%%%%%%%%%%%


\subsection{Lastnosti metrične dimenzije} \label{s:lastnosti_mdim}

Oglejmo si nekaj osnovnih ugotovitev o metrični dimenziji. Iz primera \ref{pr:mdim_pot}
lahko hitro razberemo, da metrična baza ni nujno enolično določena. Za $W$ bi lahko 
vzeli tudi vozlišče $v_n$ in prišli do enakega rezultata. 
Poiščimo sedaj najbolj splošno omejitev za metrično dimenzijo.

\begin{trditev} \label{trd:cela_resljiva}
Za povezan graf $G$, je $V(G)$ rešljiva množica. Še več, za poljubno vozlišče $v_i \in V(G)$
je $W_i = V(G) \setminus \{ v_i\}$ rešljiva množica.
\end{trditev}

\begin{dokaz}
Naj bo $G$ povezan in $|V(G)|= n$. Označimo vozlišča z $v_1, ..., v_n$.
Upoštevajoč trditev \ref{trd:nicelna_razdalja} hitro opazimo, da velja 
$\forall j \in \{ 1, 2, ... , i - 1, i + 1, ... , n\}:$ vozlišče $v_j$ ima natanko $j$-to komponento 
metrične predstavitve glede na $W_i$ enako $0$. Vozlišče $v_i$ pa je edino, ki ima vse komponente 
različne od $0.$ Sledi $\forall u, v \in V(G): u \neq v \Rightarrow r_{W_i}(v) \neq r_{W_i}(u)$, 
torej je $W_i$ rešljiva množica.
Ker je $V(G) = W_i \cup \{ v_i\},$ je tudi $V(G)$ rešljiva.
\end{dokaz}


\begin{posledica} \label{po:groba_omejitev_mdim}
    Za povezan graf $G$ velja 
    $$1 \leq \beta(G) \leq |V(G)| - 1. $$
\end{posledica}


\begin{lema} \label{lema:vozlisce_max_deg}
    Naj bo $G$ povezan graf in $|V(G)| = n,$ ter naj bo $u \in V(G)$ vozlišče stopnje $n-1.$
    Potem obstaja metrična baza v $G$, ki ne vsebuje vozlišča $u.$
\end{lema}

\begin{dokaz}
    TODO
\end{dokaz}

\begin{opomba} \label{op:zadostno_preverjanje}
    Če za neko množico $S \subseteq V(G)$ preverjamo, če je rešljiva, je dovolj preveriti metrične 
    predstavitve vozlišč $v \in V(G) \setminus S.$ Vozlišča iz $S$ bodo imela natanko eno komponento 
    vektorja enako nič. 
\end{opomba}
    

V splošnem je iskanje metrične dimenzije grafa NP-poln problem. Za nekatere vrste grafov
pa lahko najdemo eksplicitne formule za njen izračun.

\begin{trditev} \label{trd:mdim_polni_pot}
    Naj bo $G$ povezan graf in $|V(G)| = n \geq 2.$ Potem velja:
    \begin{enumerate}
        \item $G = K_n \; \Leftrightarrow \; \beta(G) = n - 1.$
        \item $G = P_n \; \Leftrightarrow \; \beta(G) = 1.$
    \end{enumerate} 
\end{trditev}

\begin{dokaz}
    Implikacijo v desno stran za obe točki smo že pokazali v primerih \ref{pr:mdim_pot} in 
    \ref{pr:mdim_poln}.
    \begin{enumerate}
        \item $\Leftarrow$  TODO
        \item $\Leftarrow$
        Recimo, da imamo povezan graf $G$ na $n$ vozliščih z $\beta(G) = 1.$ Sledi, da obstaja neka 
        rešljiva baza $W = \{ w \}.$ Označimo $V(G) = \{ v_1, v_2, ... , v_{n-1}, w\}.$ Sedaj mora 
        veljati, da so števila 
        $$ d(v_1, w),  d(v_2, v_1), ..., d(v_{n-1}, w), d(w, w) $$
        paroma različna. Vemo $d(w, w) = 0$. Ker je $G$ povezan, mora obstajati vsaj eno vozlišče, 
        ki je sosednje z $w$. BSŠ naj bo $v_{n-1} \sim w$. Torej je $d(v_{n-1}, w) = 1$ in sledi, 
        da nobeno drugo vozlišče ni sosednje z $w$. Zopet zaradi povezanosti grafa obstaja vozlišče 
        sosednje z $v_{n-1},$ ki je različno od $w$. Recimo, da je to $v_{n-2}$, za katerega sedaj 
        velja $d(v_{n-2}, w) = 2.$ Spet je to edino takšno vozlišče. Nadaljujemo podobno, 
        dokler ne pridemo do $v_1.$ Dobimo graf $P_n.$
    \end{enumerate}
\end{dokaz}


\begin{trditev} \label{trd:mdim_spojev}
    Naj bo $n\geq 4$, potem velja:
    \begin{enumerate}
        \item $n \neq 6 \; \Rightarrow \; \beta(C_n + K_1) = 
        \Bigl \lfloor \frac{2n + 2}{5}\Bigr \rfloor$.
        \item $n \neq 6 \; \; \Rightarrow \; \beta(P_n + K_1) = 
        \Bigl \lfloor \frac{2n + 2}{5}\Bigr \rfloor$.
    \end{enumerate}
\end{trditev}

\begin{dokaz}
    TODO
\end{dokaz}
%v opombo lahko  zapišem kaj se zgodi špri n = 1, 2, 3
%zakaj za 6 ne velja?


%%%%%%%%%%%%%%%%%%%%%%%%%%%%%%%%%%%%%%%%%%%%%%%%%%%%%%%%%%%%%%%%%%%%%%%%%%%%%%%


\subsubsection{Metrična dimenzija in premer grafa} \label{ss:mdim_premer}

Ni presenetljivo, da lahko najdemo povezavo med metrično dimenzijo in premerom grafa.

\begin{trditev}\label{trd:groba_meja_mdim_premer}
    Naj bo $G$ povezan graf in $|V(G)| = n$. Potem velja naslednja povezava:
    $$n \leq ({\rm diam}(G))^{\beta (G)} + \beta (G). $$
\end{trditev}

\begin{dokaz}
    Naj bo $R$ rešljiva baza grafa $G$, torej $|R| = \beta(G).$ Zanima nas, največ koliko 
    vozlišč ima lahko tak graf. Vozlišča iz množice $R$ bodo imela natanko eno 
    komponento metrične predstavitve enako nič, tako se bodo te razlikovale med sabo in od
    vseh ostalih. Če vzamemo vozlišče $v \notin R$, pa velja sledeče:
    $$\forall r_i \in R: 1 \leq d(v, r_i) \leq {\rm diam}(G).$$
    
    Vseh možnih različnih metričnih predstavitev za vozlišča izven rešljive množice $R$ 
    je tako $({\rm diam}(G))^{\beta (G)}$
    in lahko zapišemo:
    $$n \leq ({\rm diam}(G))^{\beta (G)} + \beta (G).$$
\end{dokaz}

V resnici lahko red grafa z dano metrično dimenzijo in premerom še bolj omejimo.

\begin{trditev} \label{trd:meja_mdim_premer}
    Naj bo $G$ povezan graf in $|V(G)| = n$. Označimo $ \delta = {\rm diam}(G)$ in 
    $\beta = \beta (G)$. Potem velja

    $$n \leq \Bigl ( \Bigl \lfloor {\frac{2 \delta}{3}}\Bigr \rfloor + 1 \Bigr )^{\beta} + 
    \beta \sum_{i = 1}^{\lceil \delta /3 \rceil} {(2i - 1)^{\beta - 1}}. $$
\end{trditev}

\begin{dokaz}
    TODO
\end{dokaz}

Ta zgornja meja postane še bolj natančna za posamezne družine grafov, vendar v tem delu
tega ne bomo obravnavali tako podrobno.


%%%%%%%%%%%%%%%%%%%%%%%%%%%%%%%%%%%%%%%%%%%%%%%%%%%%%%%%%%%%%%%%%%%%%%%%%%%%%%%


\subsubsection{Dvojčki in metrična dimenzija} \label{ss:dvojcki_mdim}

Vpeljimo ekvivalenčno relacijo na vozliščih:
\begin{equation}\label{eq:dvojcki}
v \equiv u \Leftrightarrow N(v)\setminus \{u\} = N(u) \setminus \{v\}.
\end{equation}
Če sta vozlišči v tej ekvivalenčni relaciji, pravimo, da sta dvojčka. 
Ekvivalenčni razred vozlišča $v$ označimo z $v^{*}$, 
množico vseh ekvivalenčnih razredov s $\tau (G)$, število vseh razredov pa naj bo označeno 
z $\iota(G) = |\tau(G)|.$


\begin{lema} \label{lema:dvojcki_razdalje}
    Naj bosta $u, v \in v(G)$ dvojčka. Potem je 
    $$\forall w \in V(G) \setminus \{u, v\} : d(u, w) = d(v, w).$$
\end{lema}

\begin{dokaz}
    Naj bosta $u$ in $v$ dvojčka v grafu $G$. Označimo $V(G) = \{u, v, w_1, ..., w_k\}$ 
    in $S = N(v)\setminus \{u\} = N(u) \setminus \{v\}$. Izberimo vozlišče 
    $w_i \in V(G) \setminus \{u, v\}.$
    \begin{enumerate}
        \item $w_i \in S \; \Rightarrow \; d(u, w_i) = d(v, w_i) = 1.$
        \item $w_i \notin S \; \Rightarrow \; d(u, w_i) = m \geq 2$. 
    
        Denimo $m=2.$ Potem obstaja $w_j \in S,$ da je $w_j \sim w_i$ in sledi $d(v, w_i) = 2.$
        
        Naj bo sedaj $m > 2.$ Obstaja vozlišče $w_j,$ sosednje od $w_i$, za katerega velja 
        $d(u, w_j) = m-1.$ Potem je po indukcijski predpostavki tudi $d(v, w_j) = m-1$ in 
        sledi $d(v, w_i) = m-1 + 1 = m.$
    \end{enumerate}
    %znak za konec dokaza višje
\end{dokaz}

Iz tega sledi, da mora vsaka rešljiva množica vsebovati vsaj enega od dvojčkov.
Zapišemo lahko naslednjo trditev:

\begin{trditev} \label{trd:meja_mdim_dvojcki}
    Za povezan graf $G$ velja
    $$\beta(G) \geq \sum_{v^{*} \in \tau(G)} (|v^{*}| - 1).$$
\end{trditev}

\begin{dokaz}
    Vzemimo ekvivačenčni razred $v^{*} \in \tau(G)$.
    Po \ref{lema:dvojcki_razdalje} vidimo, da mora metrična baza vsebovati vse razen 
    največ enega elementa $v^{*}$. V nasprotnem primeru bi imeli tisti, ki niso vsebovani v rešljivi 
    bazi med seboj enake metrične predstavitve.
    To velja za vse ekvivalenčne razrede, neenačba sledi. 
\end{dokaz}


%%%%%%%%%%%%%%%%%%%%%%%%%%%%%%%%%%%%%%%%%%%%%%%%%%%%%%%%%%%%%%%%%%%%%%%%%%%%%%%

\subsubsection{Metrična dimenzija in sosedska dimenzija} \label{ss:mdim_sdim}

\begin{trditev} \label{trd:meja_sdim}
    Za poljuben graf $G$ velja
    $$\beta(G + K_1) - 1 \leq \mu(G) \leq \beta(G + K_1).$$
    Velja še več, 
    $\mu(G) = \beta(G + K_1) \Leftrightarrow$ obstaja sosedska baza $S$ grafa $G$, 
    da nobeno vozlišče ni sosednje vsem vozliščem iz $S$.
\end{trditev}

\begin{dokaz}
    TODO
\end{dokaz}


\begin{trditev} \label{trd:sdim_pot_cikel}
    Če je $n \geq 4,$ velja $\mu(C_n) = \mu(P_n) = \Bigl \lfloor \frac{2n + 2}{5}\Bigr \rfloor. $
\end{trditev}

\begin{dokaz}
    Opazimo, da velja ${\rm diam}(P_n + K_1) = {\rm diam}(C_n + K_1) = 2.$ Vozlišča so namreč sosednja, ali 
    pa najdemo sprehod dolžine dva preko vozlišča, ki pripada $K_1$. Po \ref{trd:mdim_spojev} velja 
    $$\beta(P_n + K_1) = \Bigl \lfloor \frac{2n + 2}{5}\Bigr \rfloor$$
    Če se spomnimo še druge točke \ref{trd:lastnosti_sdim} sledi 
    $$\mu(P_n + K_1) = \beta(P_n + K_1) = \Bigl \lfloor \frac{2n + 2}{5}\Bigr \rfloor.$$
    Spojitev dodatnega vozlišča s potjo ne vpliva na sosednost vozlišč v grafu $P_n,$ zato velja
    $$\mu(P_n) \leq \mu(P_n + K_1) \leq \mu(P_n) + K_1.$$
    Na novo spojeno vozlišče, je sosednje z vsemi vozlišči v poti. Enakost z zgornjo mejo torej velja natanko
    tedaj, ko ima eno od vozlišč v poti sosedsko predstavitev enako vektorju samih enic, sicer je 
    $\mu(P_n + K_1) = \beta(P_n + K_1).$
    
    V poti je vsako vozlišče sosedno kvečjemu dvem ostalim, torej imamo lahko vektor samih enic samo,
    če je $\mu(P_n) = 2$ - manj ni, saj je $n \geq 4$, glej \ref{trd:lastnosti_sdim}, točko 4.
    Denimo, da je $s_W(u_i) = (1, 1).$ To pomeni, da je $W =\{ u_{i-1}, u_{u+1}\}.$  
    TODO
    %LAHKO POKAŽEŠ DA FORMULA NVELJA ZA N=4,5,6 NA ROKE IN NATO Z INDUKCIJO OD 7 NAPREJ DA NE MORE BITI 
    %VOZLIŠČA S PREDSTAVITVIJO (1,1)
    % BI MORDA KORISTILA KAKA LEMA GLEDE DIMENZIJE PODGRAFA.?
\end{dokaz}


\begin{trditev}
    Naj bo $K_{m_1, ..., m_t}$ $t$-delni polni graf, v katerem ima $r$ delov vsaj $2$ vozlišči, 
    ter naj velja $\sum_{i=1}^{t} m_i = m.$ Potem je
    $$\mu(K_{m_1, ..., m_t}) = \beta(K_{m_1, ..., m_t}) = 
    \begin{cases}
        m - r - 1; & r \neq t \\
        m - r; & r = t 
    \end{cases}.$$
\end{trditev}

\begin{dokaz}
    TODO
\end{dokaz}



%%%%%%%%%%%%%%%%%%%%%%%%%%%%%%%%%%%%%%%%%%%%%%%%%%%%%%%%%%%%%%%%%%%%%%%%%%%%%%%
%%%%%%%%%%%%%%%%%%%%%%%%%%%%%%%%%%%%%%%%%%%%%%%%%%%%%%%%%%%%%%%%%%%%%%%%%%%%%%%
%%%%%%%%%%%%%%%%%%%%%%%%%%%%%%%%%%%%%%%%%%%%%%%%%%%%%%%%%%%%%%%%%%%%%%%%%%%%%%%
%%%%%%%%%%%%%%%%%%%%%%%%%%%%%%%%%%%%%%%%%%%%%%%%%%%%%%%%%%%%%%%%%%%%%%%%%%%%%%%


\section{Leksikografski produkt grafov}\label{s:leks_prod}


\begin{definicija} \label{def:leks_prod}
    Leksikografski produkt $G[H]$ grafov $G$ in $H$ je definiran na množici vozlišč 
    $V (G[H]) = V (G)\times V (H)$. Dve različni vozlišči $(u, v)$ in $(x, y)$ sta 
    sosedni, kadar velja
\begin{itemize}
    \item $ux \in E(G)$ ali
    \item $u = x$ in $vy \in E(H).$ 
\end{itemize}
\end{definicija}


%%%%%%%%%%%%%%%%%%%%%%%%%%%%%%%%%%%%%%%%%%%%%%%%%%%%%%%%%%%%%%%%%%%%%%%%%%%%%%%
%%%%%%%%%%%%%%%%%%%%%%%%%%%%%%%%%%%%%%%%%%%%%%%%%%%%%%%%%%%%%%%%%%%%%%%%%%%%%%%


\subsection{Primer} 
Za lažjo predstavo si lahko ogledamo sliko \ref{fig:produkt}, ki prikazuje leksikografski 
produkt dveh naključnih povezanih grafov.

\begin{figure}[h]
    \centering
    \includegraphics[width=\textwidth]{IMG_produkt.jpg}   
    \caption{Leksikografski produkt povezanih grafov $G$ in $H$.}   
    \label{fig:produkt}
\end{figure}


%%%%%%%%%%%%%%%%%%%%%%%%%%%%%%%%%%%%%%%%%%%%%%%%%%%%%%%%%%%%%%%%%%%%%%%%%%%%%%%
%%%%%%%%%%%%%%%%%%%%%%%%%%%%%%%%%%%%%%%%%%%%%%%%%%%%%%%%%%%%%%%%%%%%%%%%%%%%%%%


\subsection{Lastnosti} \label{ss:lastnosti_leks_prod}
Nekaj osnovnih lastnosti leksikografskega produkta grafov:
\begin{itemize}
    \item RED: $|V(G)| = n$ in $|V(H)| = m \; \Rightarrow |V(G[H])| = n \cdot m.$
    \item POVEZANOST: $G[H]$ je povezan $\Leftrightarrow$ $G$ povezan. 
    \item NEKOMUTATIVNOST: v splošnem velja $G[H] \neq H[G].$
    \item DISTRIBUTIVNOST: $(G_1 + G_2)[H] = G_1[H] + G_2[H],$ 
    \item ENAKOST KOMPLEMENTOV: $\overline{G[H]} = \overline{G} [\overline{H}].$
    % kako bi se v resnici pravilno reklo tej lastnosti...?
    \item PREMER: $$ {\rm diam}(G[H]) =  \begin{cases} 
        {\rm diam}(G); & |V(G)| \geq 2 \\
        {\rm diam}(G); & G = K_1
        \end{cases}$$

\end{itemize}

Poglejmo si, kako izgleda razdalja med vozliščema v leksikografskem produktu grafov. 
Opazujemo leksikografski produkt povezanega grafa $G$ reda $n$, z množico vozlišč
$V(G) = \{v_1, v_2, ... , v_n \}$ in grafa $H$ reda $m$, z množico vozlišč 
$V(H) = \{u_1, u_2, ... , u_m \}$. Vpeljimo oznako 
$v_{ij} := (v_i, u_j) \in V(G[H]).$
Sedaj lahko zapišemo
\begin{equation} \label{eq:razdalja_produkta}
    d_{G[H]}(v_{ij}, v_{rs}) = 
    \begin{cases}
        d_G(v_i, v_r); & v_i \neq v_r \\
        a_H(u_j, u_s); & \text{sicer}
    \end{cases}
\end{equation} 

Tu je preslikava $a$ definirana v \ref{eq:fja_a}.


%%%%%%%%%%%%%%%%%%%%%%%%%%%%%%%%%%%%%%%%%%%%%%%%%%%%%%%%%%%%%%%%%%%%%%%%%%%%%%%
%%%%%%%%%%%%%%%%%%%%%%%%%%%%%%%%%%%%%%%%%%%%%%%%%%%%%%%%%%%%%%%%%%%%%%%%%%%%%%%
%%%%%%%%%%%%%%%%%%%%%%%%%%%%%%%%%%%%%%%%%%%%%%%%%%%%%%%%%%%%%%%%%%%%%%%%%%%%%%%
%%%%%%%%%%%%%%%%%%%%%%%%%%%%%%%%%%%%%%%%%%%%%%%%%%%%%%%%%%%%%%%%%%%%%%%%%%%%%%%


\section{Metrična dimenzija leksikografskega produkta grafov}\label{s:mdim_prod}
 

%%%%%%%%%%%%%%%%%%%%%%%%%%%%%%%%%%%%%%%%%%%%%%%%%%%%%%%%%%%%%%%%%%%%%%%%%%%%%%%
%%%%%%%%%%%%%%%%%%%%%%%%%%%%%%%%%%%%%%%%%%%%%%%%%%%%%%%%%%%%%%%%%%%%%%%%%%%%%%%


\subsection{Metrična dimenzija leksikografskega produkta glede na metrično 
dimenzijo grafa $H$} \label{ss:mdim_komp_prod}

%TA PODNASLOV NI NAJBOLJ POSREČEN...
V tem razdelku obravnavamo leksikografski produkt $G[H]$, kjer je $G$ povezan graf in $H$ 
poljuben graf. Naj bosta $a \in V(G)$ in $b \in V(H)$ poljubni vozlišči. Za potrebe tega 
podpoglavlja vpeljimo naslednje oznake:
\begin{itemize}
    \item $H(a) = \{ (a, v) \; | \; v \in V(H) \}$.
    \item $G(b) = \{ (v, b) \; | \; v \in V(G) \}$.
    \item Če so $H_1, H_2, \ldots , H_k$ komponente grafa $H$, označimo 
    $H_i(a) = \{ (a, v) \; | \; v \in V(H_i) \}$.
\end{itemize}

Vzemimo sosednji vozlišči $a$, $b \in V(G)$. Vemo, da je vsako vozlišče iz $H(b)$ 
sosednje vsakemu iz $H_i(a).$ 
Hitro lahko preverimo, da je inducirani podgraf grafa $G[H]$, kjer vzamemo eno 
vozlišče iz množice $H(b)$ in vsa vozlišča iz $H_i(a)$, izomorfen grafu $H_i + K_1.$ 
V nadaljevanju bomo pokazali, da lahko z rešljivo bazo tega združenega grafa
omejimo rešljivo bazo $G[H]$.

TODO - leme

% \begin{lema}
%     Naj bo $Q$ povezan graf. Obstaja rešljiva baza $S$ grafa $Q + K_1,$ da je 
%     $S \subseteq V(Q).$
% \end{lema}

% \begin{dokaz}
%     Združitev grafov $Q + K_1$ ima množico vozlišč $V(Q) \cup \{ v \}.$ Naj bo $S$ 
%     rešljiva baza $Q + K_1$. Če $v \notin S.$ je lema dokazana. Denimo $v \in S.$ 
%     V tem primeru ločimo dve situaciji:
%     \begin{enumerate}
%         \item $S \setminus \{ v \} = \emptyset$ 
        
%         Trditev \ref{trd:mdim_polni_pot} nam pove $Q = P_1 = K_1.$ Sledi $Q + K_1 = P_2.$ 
%         Hitro lahko preverimo, da je rešljiva baza grafa $P_2$ množica, ki vsebuje enega 
%         od robnih vozlišč. Torej bi lahko namesto $v$, vzeli vozlišče, ki sestavlja graf 
%         $Q$ in tako dobimo rešljivo bazo $S \subseteq V(Q).$
        
%         \item $S \setminus \{ v \} \not = \emptyset$
        
%         Definirajmo $B = V(Q + K_1) \setminus S$ in naj bo $r = |B|,$ torej 
%         $B = \{b_1, b_2, ..., b_r\}$. Za $t \in \{1, 2, ... r\}$ definiramo množici 
%         $S_t = S \cup \{b_t\}$ in $B_t = B \setminus \{b_t\}.$
%         Če obstaja $t$, da $\forall u \in B_t : \; r_{S_t}(u) \not = (1, 1, ..., 1),$
%         je lema dokazana. Sicer je $Q + K_1$ polen graf. Za poln graf pa lahko vzamemo
%         rešljivo množico, ki vsebuje vse razen enega vozlišča, torej $S = V(Q).$
%     \end{enumerate}
% \end{dokaz}

%NEKAJ LEM VMES


%%%%%%%%%%%%%%%%%%%%%%%%%%%%%%%%%%%%%%%%%%%%%%%%%%%%%%%%%%%%%%%%%%%%%%%%%%%%%%%


\subsubsection{H je nepovezan graf} \label{sss:nepovezan}


\begin{izrek} \label{izrek:omejitve_mdim_komp}
    Naj bo $G$ povezan graf reda $n \geq 2$ in $H$ poljuben graf reda $m \geq 2$, s $k \geq 1$ 
    komponentami $H_1, H_2, ... , H_k$. Potem velja:
    $$
    n \cdot \Bigl ( \Bigl ( \sum_{p=1}^{k} \beta(H_p) \Bigr )  - 1  \Bigr ) 
    \leq \beta(G[H]) \leq 
    n \cdot \Bigl ( \Bigl ( \sum_{p=1}^{k} \beta(H_p + K_1) \Bigr ) + k - 1  \Bigr ) + (n-2). 
    $$
    \end{izrek}
    % A BI MORAL IMET V TEM IZREKU H VSAJ DVE KOMPONENTI?
    
    
    V dokazu naslednjega izreka bomo konstruirali grafe, katerih metrična dimenzija je enaka
    spodnji ali zgornji meji iz izreka \ref{izrek:omejitve_mdim_komp} ter dvema vmesnima vrednostima.
    
    \begin{izrek} \label{izrek:primeri_mdim_komp}
    Obstajata taka grafa $G$ in $H$, da je $G$ povezan graf reda $n \geq 2$ in $H$ poljuben graf 
    reda $m \geq 2$, s $k \geq 1$ komponentami $H_1, H_2, \dots , H_k$, da velja:
    \begin{enumerate}
        \item $\beta(G[H]) = n \cdot \Bigl ( \Bigl ( \sum_{p=1}^{k} \beta(H_p) \Bigr )  - 1  \Bigr )$.
        \item $\beta(G[H]) = n \cdot \Bigl ( \Bigl ( \sum_{p=1}^{k} \beta(H_p + K_1) \Bigr ) + k - 1 
         \Bigr ) + (n-2)$.
        \item $\beta(G[H]) = n \cdot \Bigl ( \Bigl ( \sum_{p=1}^{k} \beta(H_p + K_1) \Bigr ) + k - 1  
        \Bigr )$.
        \item $\beta(G[H]) = n \cdot \Bigl ( \sum_{p=1}^{k} \beta(H_p + K_1) \Bigr ) $.
    \end{enumerate}
    \end{izrek}
    
    \begin{dokaz}
        \begin{enumerate}
            \item Naj bo $G = P_n$, $n\geq 4$ in $H = N_k$, $k\geq 2$. 
            Zaradi izreka \ref{izrek:omejitve_mdim_komp} je dovolj pokazati 
            $\beta(G[H]) \leq n \cdot \Bigl ( \Bigl ( \sum_{p=1}^{k} \beta(H_p) \Bigr )  - 1  \Bigr ) = n 
            \cdot (k - 1)$. Označimo $V(G) = \{p_1 , p_2, ..., p_n\}$, kjer so $\forall 1 \leq i < n : \; 
            p_i p_{i+1} \in E(G)$, in $V(H) = \{ v_1, v_2, ..., v_k\}.$ Definirajmo množico 
            $W = V(G[H]) \setminus G(v_k).$ Velja $|W| = n \cdot (k-1).$ 
            Pokažimo, da je $W$ rešljiva množica. Opomba \ref{op:zadostno_preverjanje} nam pove, da je dovolj preveriti
            vozlišča iz množice $G(v_k) = \{ (p_1, v_k), (p_2, v_k), ..., (p_n, v_k)\}$. Če se spomnimo formule
            \ref{eq:razdalja_produkta}, vidimo, da velja:
            \begin{itemize}
                \item $2 \leq d((p_i, v_k), (p_{j+1}, v_1)) \neq d((p_j, v_k), (p_{j + 1}, v_1)) = 1,$ 
                za $ 1 \leq i \leq j < n $.
                \item $2 \leq d((p_n, v_k), (p_{i-1}, v_1)) \neq d((p_i, v_k), (p_{i- 1}, v_1)) = 1,$ 
                za $ 2 \leq i < n $.
                \item $1 = d((p_1, v_k), (p_2, v_1)) \neq d((p_n, v_k), (p_2, v_1)) \geq 2.$
            \end{itemize}
            Sledi, da so metrične predstavitve vozlišč iz $G(v_k)$ paroma različne in je $W$ rešljiva množica.
    
            \item Naj bo $G = S_{n-1}$ zvezda na $n$ vozliščih, $n\geq 4$, in $H$ graf s $k \geq 2$ komponentami
            $H_1, H_2, ..., H_k,$ kjer je $H_i = P_8.$ Velja $\beta(P_8 + K_1) = 
            \Bigl \lfloor \frac{2\cdot 8 + 2}{5} \Bigr \rfloor = 3. $ % DOKAZ??
            Zato je, podobno kot v prvi točki, dovolj pokazati  
            $$\beta(G[H]) \geq n \cdot \Bigl ( \Bigl ( \sum_{p=1}^{k} \beta(H_p + K_1) \Bigr ) + k - 1 \Bigr ) + 
            (n-2) = 4kn - 2.$$ 
            Naj bo $B$ rešljiva baza $P_8 + K_1$, potem obstaja $v \in V(P_8),$ da je $r_B(v) = (2, 2, 2).$
            %DOKAZ??
    
            Recimo, da je $\beta(G[H]) < 4kn - 2,$ in naj bo $W$ rešljiva baza $G[H]$. 
            
            TODO
    
            \item TODO
            \item TODO
    
        \end{enumerate}
    \end{dokaz}


%%%%%%%%%%%%%%%%%%%%%%%%%%%%%%%%%%%%%%%%%%%%%%%%%%%%%%%%%%%%%%%%%%%%%%%%%%%%%%%


\subsubsection{H je povezan graf} \label{sss:povzean}


\begin{izrek} \label{izrek:primeri_mdim_komp_povezan}
    Naj bo $G$ povezan graf reda $n \geq 2$ in $H$ povezan graf reda $m \geq 2$. Potem velja:
    $$
    n \cdot \beta(H)  
    \leq \beta(G[H]) \leq 
    n \cdot \beta(H_p + K_1) + (n-2). 
    $$
\end{izrek}

\begin{izrek} \label{izrek:omejitve_mdim_komp_povezan}
    Obstajata taka grafa povezana $G$ in $H$, reda vsaj $2$, da velja:
    \begin{enumerate}
        \item $\beta(G[H]) = n \cdot \beta(H)$
        \item $\beta(G[H]) = n \cdot \beta(H_p + K_1) + (n-2)$
        \item $\beta(G[H]) = n \cdot \beta(H_p + K_1)$
    \end{enumerate}
\end{izrek}
    
\begin{dokaz}
    TODO
\end{dokaz}

Na tej točki se lahko vprašamo, če za vsako vrednost $c$ znotraj zgornjih mej lahko najdemo
grafa $G$ in $H$, ki bosta zadoščala $\beta(G[H]) = c$.

%%%%%%%%%%%%%%%%%%%%%%%%%%%%%%%%%%%%%%%%%%%%%%%%%%%%%%%%%%%%%%%%%%%%%%%%%%%%%%%
%%%%%%%%%%%%%%%%%%%%%%%%%%%%%%%%%%%%%%%%%%%%%%%%%%%%%%%%%%%%%%%%%%%%%%%%%%%%%%%


\subsection{Metrična dimenzija, sosedska dimenzija in dvojčki}
V tem podpoglavju bomo obravnavali metrično dimenzijo leksikografskega produkta $G[H]$ na 
podlagi reda grafa $G$ in sosedske dimenzije grafa $H$. 

\begin{lema}
    Naj bo $G$ povezan graf reda $n$ in $H$ poljuben graf. Potem velja
    $\beta(G[H]) \geq n \cdot \mu(H).$
\end{lema}

\begin{dokaz}
    TODO
\end{dokaz}


\begin{trditev}
    Naj bo $G$ povezan graf reda $n$ in $H$ poljuben graf. Če obstajata dve sosedski bazi 
    grafa $H$, $S_1$ in $S_2$, da nobeno vozlišče nima sosedske predstavitve z množico $S_1$ 
    enako $\1$ in nobeno vozlišče nima sosedske predstavitve z množico $S_2$ 
    enako $\2$, potem velja $$\beta(G[H]) = \beta(G[\overline{H}]) = n \cdot \mu(H) $$.
\end{trditev}

\begin{dokaz}
    TODO
\end{dokaz}


\begin{trditev}
    Naj bo $G$ povezan graf reda $n$ in $H$ poljuben graf. Če za vsako sosedsko bazo $S$
    grafa $H$ obstajata vozlišči s sosedskima predstavitvama glede na $S$ enakima $\1$ in $\2$,
    potem je 
    $$\beta(G[H]) = \beta(G[\overline{H}]) = n \cdot (\mu(H) + 1) - \iota(G). $$
\end{trditev}

\begin{dokaz}
    TODO
\end{dokaz}


\begin{trditev}
    Naj bo $G$ povezan graf reda $n$ in $H$ poljuben graf. Naj ima $H$ sledeči lastnosti:
    \begin{enumerate}
        \item za vsako sosedsko bazo obstaja vozlišče s sosedsko predstavitvijo $\1$,
        \item obstaja sosedska baza $S$, da nobeno vozlišče nima sosedske predstavitve enake $\2.$
    \end{enumerate} 
    Potem velja $$\beta(G[H]) = n \cdot \mu(H) + a(G) - \iota_K(G). $$
\end{trditev}

\begin{dokaz}
    TODO
\end{dokaz}


\begin{trditev}
    Naj bo $G$ povezan graf reda $n$ in $H$ poljuben graf. Naj ima $H$ sledeči lastnosti:
    \begin{enumerate}
        \item za vsako sosedsko bazo obstaja vozlišče s sosedsko predstavitvijo $\2$,
        \item obstaja sosedska baza $S$, da nobeno vozlišče nima sosedske predstavitve enake $\1.$
    \end{enumerate} 
    Potem velja $$\beta(G[H]) = n \cdot \mu(H) + b(G) - \iota_N(G). $$
\end{trditev}

\begin{dokaz}
    TODO
\end{dokaz}


\begin{izrek}
    Če je $G$ povezan graf reda $n$, ki nima dvojčkov, velja $\beta(G[H]) = n \cdot \mu(H).$
\end{izrek}

\begin{dokaz}
    Če $G$ nima dvojčkov, je $\iota(G) = n$, $\iota_K(G) = a(G) = 0$ in $\iota_N(G) = b(G) = 0.$
    Graf $H$ gotovo zadostuje pogojem v eni od zgornjih treh trditev, zato sledi  
    $\beta(G[H]) = n \cdot \mu(H).$
\end{dokaz}


\begin{izrek}
    Naj bo $G = P_n$, za $n\geq 4$ ali $G = C_n$ za $n \geq 5,$ ter naj bo $m\geq 3.$ Tedaj velja
    $\beta(G[P_m]) = \beta(G[C_m]) = \beta(G[\overline{P_m}]) = \beta(G[\overline{C_m}]) = 
    n \cdot  \Bigl \lfloor \frac{2\cdot m + 2}{5} \Bigr \rfloor $.

    Velja še več, 
    $$\beta(G[K_{m_1, ..., m_t}]) = \beta(G[\overline{K}_{m_1, ..., m_t}]) = 
    \begin{cases}
        n \cdot (m - r - 1); & r \neq t, \\
        n \cdot (m - r); & r = t, 
    \end{cases} $$
    kjer so $m_1, ..., m_r \geq 2$,  $m_{r+1} = ... = m_t = 1$ in $\sum_{i= 1}^{t} m_i = m.$
\end{izrek}

\begin{dokaz}
    TODO
\end{dokaz}

\begin{izrek}
    Naj bodo $n, m, m_1, ..., m_t$ števila, za katera velja:
    \begin{itemize}
        \item $n \geq 2$,
        \item $m_1, ..., m_r \geq 2$,
        \item $m_{r+1} = ... = m_t = 1$,
        \item $\sum_{i= 1}^{t} m_i = m.$.
    \end{itemize}
    Potem velja
    $$\beta(K_n[K_{m_1, ..., m_t}]) = 
    \begin{cases}
        n \cdot (m - r) - 1; & r \neq t, \\
        n \cdot (m - r); & r = t, 
    \end{cases} $$.
\end{izrek}

\begin{dokaz}
    TODO
\end{dokaz}




%%%%%%%%%%%%%%%%%%%%%%%%%%%%%%%%%%%%%%%%%%%%%%%%%%%%%%%%%%%%%%%%%%%%%%%%%%%%%%%
%%%%%%%%%%%%%%%%%%%%%%%%%%%%%%%%%%%%%%%%%%%%%%%%%%%%%%%%%%%%%%%%%%%%%%%%%%%%%%%
%%%%%%%%%%%%%%%%%%%%%%%%%%%%%%%%%%%%%%%%%%%%%%%%%%%%%%%%%%%%%%%%%%%%%%%%%%%%%%%
%%%%%%%%%%%%%%%%%%%%%%%%%%%%%%%%%%%%%%%%%%%%%%%%%%%%%%%%%%%%%%%%%%%%%%%%%%%%%%%


\section{Zaključek}
% ...
\printbibliography

\end{document}
