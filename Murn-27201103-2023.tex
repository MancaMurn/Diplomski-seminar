\documentclass[mat1, tisk]{fmfdelo}
\usepackage{amsmath}
% \documentclass[fin1, tisk]{fmfdelo}
% Če pobrišete možnost tisk, bodo povezave obarvane,
% na začetku pa ne bo praznih strani po naslovu, …

%%%%%%%%%%%%%%%%%%%%%%%%%%%%%%%%%%%%%%%%%%%%%%%%%%%%%%%%%%%%%%%%%%%%%%%%%%%%%%%
% METAPODATKI
%%%%%%%%%%%%%%%%%%%%%%%%%%%%%%%%%%%%%%%%%%%%%%%%%%%%%%%%%%%%%%%%%%%%%%%%%%%%%%%

% - vaše ime
\avtor{Manca Murn}

% - naslov dela v slovenščini
\naslov{Metrična dimenzija leksikografskega produkta grafov}

% - naslov dela v angleščini
\title{The metric dimension of the lexicographic product of graphs}

% - ime mentorja/mentorice s polnim nazivom:
%   - doc.~dr.~Ime Priimek
%   - izr.~prof.~dr.~Ime Priimek
%   - prof.~dr.~Ime Priimek
%   za druge variante uporabite ustrezne ukaze
\mentor{Sandi Klavžar}
% \somentor{...}
% \mentorica{...}
% \somentorica{...}
% \mentorja{...}{...}
% \somentorja{...}{...}
% \mentorici{...}{...}
% \somentorici{...}{...}

% - leto diplome
\letnica{2023} 

% - povzetek v slovenščini
%   V povzetku na kratko opišite vsebinske rezultate dela. Sem ne sodi razlaga
%   organizacije dela, torej v katerem razdelku je kaj, pač pa le opis vsebine.
\povzetek{...}

% - povzetek v angleščini
\abstract{...}

% - klasifikacijske oznake, ločene z vejicami
%   Oznake, ki opisujejo področje dela, so dostopne na strani https://www.ams.org/msc/
\klasifikacija{..., ...}

% - ključne besede, ki nastopajo v delu, ločene s \sep
\kljucnebesede{...\sep ...}

% - angleški prevod ključnih besed
\keywords{...\sep ...} % angleški prevod ključnih besed

% - angleško-slovenski slovar strokovnih izrazov
\slovar{
% \geslo{angleški izraz}{slovenski izraz}
% ...
}

% - ime datoteke z viri (vključno s končnico .bib), če uporabljate BibTeX
% \literatura{....bib}

%%%%%%%%%%%%%%%%%%%%%%%%%%%%%%%%%%%%%%%%%%%%%%%%%%%%%%%%%%%%%%%%%%%%%%%%%%%%%%%
% DODATNE DEFINICIJE
%%%%%%%%%%%%%%%%%%%%%%%%%%%%%%%%%%%%%%%%%%%%%%%%%%%%%%%%%%%%%%%%%%%%%%%%%%%%%%%

% naložite dodatne pakete, ki jih potrebujete
% \usepackage{...}

% deklarirajte vse matematične operatorje, da jih bo LaTeX pravilno stavil
% \DeclareMathOperator{\...}{...}

% vstavite svoje definicije ...
% \newcommand{\...}{...}


%%%%%%%%%%%%%%%%%%%%%%%%%%%%%%%%%%%%%%%%%%%%%%%%%%%%%%%%%%%%%%%%%%%%%%%%%%%%%%%
% ZAČETEK VSEBINE
%%%%%%%%%%%%%%%%%%%%%%%%%%%%%%%%%%%%%%%%%%%%%%%%%%%%%%%%%%%%%%%%%%%%%%%%%%%%%%%

\begin{document}

\section{Uvod}
% ...
\section{Metrična dimenzija grafa}

%motivacija neki izvirnega

\subsection{Definicija}

Metrična dimenzija grafa je najmanjše število vozlišč grafa, ki jih potrebujemo, da
vsa vozlišča v grafu razlikujemo med sabo zgolj s pomočjo razdalj do izbranih vozlišč.
V matematičnem jeziku to povemo takole:

\begin{definicija}
    Naj bo $G$ povezan graf in $W = \{ w_1, ... , w_k  \} \subseteq V(G)$ neprazna podmnožica vozlišč. 
    Vektor $r_W(v) = (d(v, w_1), ..., d(v, w_k))$ imenujemo metrična predstavitev vozlišča $v \in V(G)$ s podmnožico $W$.
\end{definicija}

\begin{definicija}
    Neprazna podmnožica $R \subset V(G)$ je rešljiva,
    če $\forall u, v \in V(G): u \neq v \implies r_R(v) \neq r_R(u)$.
\end{definicija}

\begin{definicija}
    Najmanjša rešljiva množica grafa $G$ se imenuje rešljiva baza. Njeno velikost imenujemo metrična dimenzija in jo označimo z $\beta(G).$ 
\end{definicija}

Poglejmo si nekaj lahkih osnovnih primerov.

\begin{primer} \label{primer_2.4.}
Graf poti dolžine $n$ označujmo z $P_n$. Vozlišča označimo z $v_1, v_2, ..., v_n$, kot 
je prikazano na spodnji sliki. Izberimo podmnožico $W = {v_1} \subseteq V(G).$ 
Metrične predstavitve vozlišč grafa $P_n$, glede na izbrano podmnožico vozlišč, so potem sledeče:
\begin{align*}
    r_W(v_1) = d(v_1, v_1) & = 0 \\
    r_W(v_2) = d(v_2, v_1) & = 1 \\
    & \dots \\
    r_W(v_n) = d(v_n, v_1) & = n-1.
\end{align*}

Vidimo, da so metrične predstavitve vseh vozlišč med seboj različne.
Sledi, da je $W$ rešljiva množica. Ker je njena velikost enaka $1$ in je to najmanjša možna neprazna 
podmnožica vozlišč, je torej metrična dimenzija
grafa poti poljubne dolžine enaka $\beta(P_n) = 1.$
\end{primer}

\begin{primer}\label{primer_2.5.}
    Cikel dolžine $n$ označujmo z $C_n$. Vozlišča označimo z $v_1, v_2, ..., v_n$, kot 
    je prikazano na spodnji sliki. Izberimo podmnožico $W = {v_1, v_2} \subseteq V(G).$ 
    Metrične predstavitve vozlišč grafa $C_n$, glede na izbrano podmnožico vozlišč, so potem sledeče:
    \begin{align*}
        r_W(v_1) = (d(v_1, v_1), d(v_1, v_2)) & = (0, 1) \\
        r_W(v_2) = (d(v_2, v_1), d(v_2, v_2)) & = (1, 0) \\
        r_W(v_3) = (d(v_3, v_1), d(v_3, v_2)) & = (2, 1) \\
        r_W(v_4) = (d(v_4, v_1), d(v_4, v_2)) & = (3, 2) \\
        & \dots \\
        r_W(v_{n-1}) = (d(v_{n-1}, v_1), d(v_{n-1}, v_2)) & = (2, 3) \\
        r_W(v_n) = (d(v_n, v_1), d(v_n, v_2)) & = (1, 2)
    \end{align*}
    
    Zopet vidimo, da so metrične predstavitve vseh vozlišč med seboj različne.
    $W$  je torej rešljiva množica, njena velikost pa je enaka $2$. Metrična dimenzija
    poljubno velikega cikla je enaka $\beta(C_n) = 2.$
\end{primer}

\subsection{Rešljiva množica}

Nekaj osnovnih ugotovitev o rešljivih množicah grafa lahko razberemo iz zgornjih primerov.

\begin{trditev}
Za povezan graf $G$, je $V(G)$ rešljiva množica.
\end{trditev}
\begin{dokaz}
Predpostavimo, da ima graf $G$ $n$ vozlišč. Označimo vozlišča grafa $G$ z $v_1, ..., v_n$.
Za posamezno vozlišče bo metrična predstavitev sledeča:
$$r_{V(G)}(v_k) = (d(v_k, v_1), ..., d(v_k, v_k), ... , d(v_k, v_n)) = (d(v_k, v_1), ..., 0 , ... , d(v_k, v_n)).$$
Torej za vsako vozlišče $v_k$ bo $k$-ta komponenta metrične predstavitve enaka $0$. To je tudi edina komponenta v vektorju, 
ki bo enaka $0$, saj za razdaljo med vozlišči velja
$$d(v, w) = 0 \Leftrightarrow v = w.$$
Sledi $\forall u, v \in V(G): u \neq v \implies r_{V(G)}(v) \neq r_{V(G)}(u)$, torej je $V(G)$ rešljiva množica.
\end{dokaz}

% a je sploh definirana razdalja na nepovezanem grafu?? s bi se dalo narest kakšno smiselno razširitev??
%iz te trditve sledi, da metrična dimenzija vselej obstaja in je manjša ali enaka |V(G)|.

\begin{trditev}
    Rešljiva baza grafa $G$ ni enolično določena.
\end{trditev}
\begin{dokaz}
    To hitro vidimo na primeru %\ref*{primer_2.4.}.
    Za $W$ bi lahko vezli tudi vozlišče $v_2$ in prišli do enakega rezultata.
\end{dokaz}

V defnicijah 2.1. in 2.2. smo prepostavili, da imamo neprazno podmnožico vozlišč. Če bi vzeli prazno množico,
bi bila definicija očitno nesmiselna. Iz trditve 2.6. lahko potem sklepamo, da metrična dimenzija za graf vselej 
obstaja in lahko zapišemo naslednjo posledico:

\begin{posledica}
    Za povezan graf $G$ velja
    $$1 \leq \beta(G) \leq |V(G)| - 1. $$
\end{posledica}
\begin{dokaz}
    Iz definicije metrične dimenzije sledi $1 \leq \beta(G)$. Iz trditve 2.6. pa sledi $\beta(G) \leq |V(G)|.$
    Predpostavimo $|V(G)| = n$ in $V(G) = \{ v_1, ... , v_n\}$. Vzemimo sedaj podmnožico $W = \{ v_1, ... , v_{n-1}\}.$
    Metrične predstavitve vozlišč glede na $W$ so sledeče:

    \begin{align*}
        r_W(v_1) & = (0, d(v_1, v_2), ..., d(v_1, v_k), ... , d(v_1, v_{n-1})) \\
        r_W(v_2) & = (d(v_2, v_1), 0, ..., d(v_2, v_k), ... , d(v_2, v_{n-1})) \\
        & \dots \\
        r_W(v_k) & = (d(v_k, v_1), d(v_k, v_2), ..., 0 , ... , d(v_k, v_{n-1})) \\
        & \dots \\
        r_W(v_{n-1}) & = (d(v_{n-1}, v_1), d(v_{n-1}, v_2), ... , d(v_{n-1}, v_k) , ..., 0) \\
        r_W(v_n) & = (d(v_n, v_1), d(v_n, v_2), ...,  d(v_n, v_k), ... , d(v_n, v_{n-1})) \\
    \end{align*}
    Vidimo, da so vse metrične predstavitve med seboj različne, torej je $W$ rešljiva in $\beta(G) \leq n - 1.$ 




\end{dokaz}

%Konec koncev je vsega konec.%
%Začetki so najtežji.%





%%%%%%%%%%%%%%%%%%%%%%%%%%%%%%%%%%%%%%%%%%%%%%%%%%%%%%%%%%%%%
\section{Leksikografksi produkt grafov}
\section{Metrična dimenzija leksikografskega produkta grafov}
% \section{...}
% ...

% \section{Zaključek}
% ...

\end{document}
